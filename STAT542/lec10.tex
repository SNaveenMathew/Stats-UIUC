\documentclass[]{article}
\usepackage{lmodern}
\usepackage{amssymb,amsmath}
\usepackage{ifxetex,ifluatex}
\usepackage{fixltx2e} % provides \textsubscript
\ifnum 0\ifxetex 1\fi\ifluatex 1\fi=0 % if pdftex
  \usepackage[T1]{fontenc}
  \usepackage[utf8]{inputenc}
\else % if luatex or xelatex
  \ifxetex
    \usepackage{mathspec}
  \else
    \usepackage{fontspec}
  \fi
  \defaultfontfeatures{Ligatures=TeX,Scale=MatchLowercase}
\fi
% use upquote if available, for straight quotes in verbatim environments
\IfFileExists{upquote.sty}{\usepackage{upquote}}{}
% use microtype if available
\IfFileExists{microtype.sty}{%
\usepackage{microtype}
\UseMicrotypeSet[protrusion]{basicmath} % disable protrusion for tt fonts
}{}
\usepackage[margin=1in]{geometry}
\usepackage{hyperref}
\hypersetup{unicode=true,
            pdftitle={Lecture 10},
            pdfauthor={Naveen Mathew Nathan S.},
            pdfborder={0 0 0},
            breaklinks=true}
\urlstyle{same}  % don't use monospace font for urls
\usepackage{graphicx,grffile}
\makeatletter
\def\maxwidth{\ifdim\Gin@nat@width>\linewidth\linewidth\else\Gin@nat@width\fi}
\def\maxheight{\ifdim\Gin@nat@height>\textheight\textheight\else\Gin@nat@height\fi}
\makeatother
% Scale images if necessary, so that they will not overflow the page
% margins by default, and it is still possible to overwrite the defaults
% using explicit options in \includegraphics[width, height, ...]{}
\setkeys{Gin}{width=\maxwidth,height=\maxheight,keepaspectratio}
\IfFileExists{parskip.sty}{%
\usepackage{parskip}
}{% else
\setlength{\parindent}{0pt}
\setlength{\parskip}{6pt plus 2pt minus 1pt}
}
\setlength{\emergencystretch}{3em}  % prevent overfull lines
\providecommand{\tightlist}{%
  \setlength{\itemsep}{0pt}\setlength{\parskip}{0pt}}
\setcounter{secnumdepth}{0}
% Redefines (sub)paragraphs to behave more like sections
\ifx\paragraph\undefined\else
\let\oldparagraph\paragraph
\renewcommand{\paragraph}[1]{\oldparagraph{#1}\mbox{}}
\fi
\ifx\subparagraph\undefined\else
\let\oldsubparagraph\subparagraph
\renewcommand{\subparagraph}[1]{\oldsubparagraph{#1}\mbox{}}
\fi

%%% Use protect on footnotes to avoid problems with footnotes in titles
\let\rmarkdownfootnote\footnote%
\def\footnote{\protect\rmarkdownfootnote}

%%% Change title format to be more compact
\usepackage{titling}

% Create subtitle command for use in maketitle
\providecommand{\subtitle}[1]{
  \posttitle{
    \begin{center}\large#1\end{center}
    }
}

\setlength{\droptitle}{-2em}

  \title{Lecture 10}
    \pretitle{\vspace{\droptitle}\centering\huge}
  \posttitle{\par}
    \author{Naveen Mathew Nathan S.}
    \preauthor{\centering\large\emph}
  \postauthor{\par}
      \predate{\centering\large\emph}
  \postdate{\par}
    \date{9/20/2019}


\begin{document}
\maketitle

\hypertarget{choosing-knots-in-spline}{%
\subsection{Choosing knots in spline}\label{choosing-knots-in-spline}}

Knots should be data adaptive. Infinite knots =\textgreater{} model if
fit perfectly, but very high degrees of freedom. Degrees of freedom = p
= number of knots.

Let us put knots at all observed \(X_i\)s in NCS. Since degrees of
freedom is same as number of points = n, we will have a design matrix in
terms of the matrix that is \(F_{n\times n}\)

Updated loss function:
\(||Y-F_{n \times n}\beta|| + \lambda \beta^T\Omega\beta\), where
\(\lambda \beta^T\Omega\beta\) is a penalty. It becomes identical to
ridge penalty if \(Omega = I\). Let us examine this loss function:

\(argmin_{g\in W^2(a,b)}\sum_{i=1}^{n} (y_i - g(x_i))^2 + \lambda \int_{a}^{b}[g''(x)]^2 dx\)
is used to penalize the second order derivative

Let us define a Sobolev space such that
\(g \in W^2(a, b); W^2(a, b) = \{g, g'\ are\ continuous, \int_{a}^{b}\rho^2(g'')dx < \infty\).
This is a different view of model complexity compared to regularization
in linear regression: in linear regression as \(\lambda \to \infty\),
\(\beta \to 0\) which leads to a constant function estimate. Whereas, in
.

Let us consider \(\tilde g\) to be a NCS such that
\(\tilde g(X_i) = g(X_i) \forall i\) and
\(\int_{a}^{b} (\tilde g'')^2 dx \le \int_{a}^{b} (g'')^2 dx\).
Evaluating this statement: Natural cubic spline has the same values at
the points \(g(X_i) = \tilde g(X_i)\), but has lower penalty term than
any function g chosen from the Sobolev space defined above. But these
functions are not equal at all X values. Therefore, let us consider the
difference \(h(x) = g(x) - \tilde g(x)\). Therefore, \(h(x_i) = 0\) and
\(\int_a^b \tilde g''^2 + \int_a^bh''^2 + 2\int_a^b \tilde g'' h''\)

Let us examine the third term:
\(\int_a^b \tilde g'' h'' = \int_a^b\tilde gdh' = \tilde g''h'|_a^b - \int_a^b h' \tilde g^{3'}dx\)

Let us examine the first term. For NCS the function is linear at the
boundaries. Therefore, \(h'(a) = h'(b) = 0\). Let us examine the second
term: \(\int_a^b h'\tilde g^{3'}\). Since the function \(\tilde g\) is a
NCS it is a sum of cubic terms between knots (first and second order
derivatives continuous, coefficient of third order term may be
different). Therefore the second term is \$\sum\_\{i=1\}\^{}\{n-1\}
\tilde g\textsuperscript{\{3'\}\int\emph{\{x\_i\}\^{}\{x\_i+1\}h'(x)dx =
\sum}\{i=1\}}\{n-1\}\tilde g\^{}\{3'\}(X\_i+){[}h(X\_\{i+1\})-h(X\_i){]}
\$

Therefore,
\(g''(x) = \sum_{i=1}^{n} \beta_iN'_i(x) = \int_a^b[sum_{i=1}^n \beta_i N''_i(x)]^2dx = \sum_{i,j} \beta_i\beta_j N''_i(x) = \beta^T\Omega\beta\)
where \(\Omega\) is a positive definite matrix


\end{document}
